% Options for packages loaded elsewhere
\PassOptionsToPackage{unicode}{hyperref}
\PassOptionsToPackage{hyphens}{url}
%
\documentclass[
]{article}
\usepackage{lmodern}
\usepackage{amssymb,amsmath}
\usepackage{ifxetex,ifluatex}
\ifnum 0\ifxetex 1\fi\ifluatex 1\fi=0 % if pdftex
  \usepackage[T1]{fontenc}
  \usepackage[utf8]{inputenc}
  \usepackage{textcomp} % provide euro and other symbols
\else % if luatex or xetex
  \usepackage{unicode-math}
  \defaultfontfeatures{Scale=MatchLowercase}
  \defaultfontfeatures[\rmfamily]{Ligatures=TeX,Scale=1}
\fi
% Use upquote if available, for straight quotes in verbatim environments
\IfFileExists{upquote.sty}{\usepackage{upquote}}{}
\IfFileExists{microtype.sty}{% use microtype if available
  \usepackage[]{microtype}
  \UseMicrotypeSet[protrusion]{basicmath} % disable protrusion for tt fonts
}{}
\makeatletter
\@ifundefined{KOMAClassName}{% if non-KOMA class
  \IfFileExists{parskip.sty}{%
    \usepackage{parskip}
  }{% else
    \setlength{\parindent}{0pt}
    \setlength{\parskip}{6pt plus 2pt minus 1pt}}
}{% if KOMA class
  \KOMAoptions{parskip=half}}
\makeatother
\usepackage{xcolor}
\IfFileExists{xurl.sty}{\usepackage{xurl}}{} % add URL line breaks if available
\IfFileExists{bookmark.sty}{\usepackage{bookmark}}{\usepackage{hyperref}}
\hypersetup{
  pdftitle={Teste de conhecimento Cinnecta},
  pdfauthor={Claudio Resende},
  hidelinks,
  pdfcreator={LaTeX via pandoc}}
\urlstyle{same} % disable monospaced font for URLs
\usepackage[margin=1in]{geometry}
\usepackage{graphicx,grffile}
\makeatletter
\def\maxwidth{\ifdim\Gin@nat@width>\linewidth\linewidth\else\Gin@nat@width\fi}
\def\maxheight{\ifdim\Gin@nat@height>\textheight\textheight\else\Gin@nat@height\fi}
\makeatother
% Scale images if necessary, so that they will not overflow the page
% margins by default, and it is still possible to overwrite the defaults
% using explicit options in \includegraphics[width, height, ...]{}
\setkeys{Gin}{width=\maxwidth,height=\maxheight,keepaspectratio}
% Set default figure placement to htbp
\makeatletter
\def\fps@figure{htbp}
\makeatother
\setlength{\emergencystretch}{3em} % prevent overfull lines
\providecommand{\tightlist}{%
  \setlength{\itemsep}{0pt}\setlength{\parskip}{0pt}}
\setcounter{secnumdepth}{-\maxdimen} % remove section numbering

\title{Teste de conhecimento Cinnecta}
\author{Claudio Resende}
\date{09/10/2020}

\begin{document}
\maketitle

\hypertarget{introduuxe7uxe3o}{%
\section{Introdução}\label{introduuxe7uxe3o}}

Este documento apresenta uma análise descritiva e preditiva de dados de
acomodações do AirBnB. A base de dados fornecida contém 34 variáveis
(colunas) e 7146 observações (linhas).

Para a análise aqui realizada foram selecionadas 22 colunas: foram
excluídas colunas identificadoras da acomodação, como latitude e
longitude, e as colunas \emph{booleanas} relacionadas às avaliações
(colunas '\_na').

A primeira parte da análise consiste em explorar as variáveis para
identificar eventuais padrões, tendências, vieses e outros tipos de
comportamento dos dados que possam requerer transformações. Em seguida,
são propostos modelos estatísticos para analisar a relação entre as
variáveis.

\hypertarget{anuxe1lise-descritivaexploratuxf3ria}{%
\subsection{Análise
descritiva/exploratória}\label{anuxe1lise-descritivaexploratuxf3ria}}

A seguir as variáveis da base de dados são analisadas individualmente e,
em seguinda, em combinação com uma ou duas outras variáveis.

\hypertarget{anfitriuxe3o}{%
\paragraph{Anfitrião}\label{anfitriuxe3o}}

O tipo de anfitrião é uma variável importante nos serviços oferecidos
pelo AirBnB porque indicam o nível de experiência do anfitrião e a
qualidade do serviço prestado. A base de dados analisada está
distribuída em 59\% e 41\% para anfitriões comuns (host) e
superanfitriões (superhost), respectivamente.

\end{document}
